%!TEX TS-program = xelatex

\documentclass[c, dvipsnames]{beamer}  % [t], [c], или [b] --- вертикальное 
%\documentclass[handout, dvipsnames, c]{beamer} % Раздаточный материал (на слайдах всё сразу)
\input{preamble_slides}
\title[OLG]{Модели пересекающихся поколений (OLG)}


\author[Михаил Гареев]{Михаил Гареев \\ \smallskip \scriptsize ЭО-15-01 \\ \smallskip \scriptsize \href{mailto:mkhlgrv@gmail.com}{\nolinkurl{mkhlgrv@gmail.com} }}

\superviser{к.э.н. Полбин А.В.}

\institute[РАНХиГС]{ \uppercase{
 Российская Академия Народного Хозяйства и  \\ Государственной Службы при Президенте Российской Федерации}}
\date{2020}


\titlegraphic{\includegraphics[scale=0.5]{logo/logo_ranepa.png}}
\titlegraphicii{\includegraphics[scale=0.5]{logo/logo_emit.png}}

\begin{document}

\frame[plain]{\titlepage}	% Титульный слайд


\section{Возникновение и основная идея}

\begin{frame}
 \frametitle{\insertsection}
    \begin{itemize}
        \item Проблема: модели экономического роста первого поколения (модель Солоу) предполагают экзогенную норму сбережения.
        \item Решение: модель, в которой сбережения формируются исходя из выбора экономических агентов:
        \begin{itemize}
        \item модель Рамсея --- Кааса --- Купманса (1963): бесконечно живущий агент выбирает уровень потребления и сбережения, максимизируя полезность;
        \item модель пересекающихся поколений Даймонда (1965): в каждый момент времени существует несколько поколений, молодые работают и сберегают, пожилые тратят свои сбережения. 
        \end{itemize}
    \end{itemize}
\end{frame}

\section{Базовая модель 2-х поколений}
\subsection{Предположения}
\begin{frame}
 \frametitle{\insertsection}
 \framesubtitle{\insertsubsection}
     \begin{itemize}
        \item дискретное время,
        \item индивиды живут два периода,
        \item полезность индивида $U_t = \ln(c_{1,t}) + \beta \ln(c_{2, t+1 })$,
        \begin{itemize}
            \item $c_{1,t}$ --- потребление молодого поколения,
        \item $c_{2,t + 1}$ --- потребление пожилого поколения,
        \item $\beta \in (0, 1)$ --- коэффициент временных предпочтений,
        \end{itemize}
        \item численность населения растёт с постоянным темпом $n$: $L_t = (1+n)^t L_0$,
        \item индивиды работают только в первый период жизни, неэластично предлагая единицу труда и получая зарплату $w_t$,
        
    \end{itemize}

\end{frame}

\begin{frame}
 \frametitle{\insertsection}
 \framesubtitle{\insertsubsection}
     \begin{itemize}
     \item выпуск описывается функцией Кобба-Дугласа: $Y_t =F(K_t, L_t) = {K_t}^\alpha {L_t}^{1-\alpha}$
        \item рынки факторов производства находятся в совершенной конкуренции,
        \item обозначим $k_t = K_t/L_t$, $f(k_t) = F(K_t, 1)$, тогда рентная цена капитала, равная доходности сбережений $R_t = 1-\delta + f^{'}(k_t)$, где $\delta$ --- амортизация,
        \item зарплата $w_t = f(k_t) - R_t k_t$,
        \item единица сбережений $s_t$ без потерь трансформируется в единицу капитала $K_t$.
    \end{itemize}

\end{frame}

\subsection{Задача потребителя}
\begin{frame}
 \frametitle{\insertsection}
 \framesubtitle{\insertsubsection}
 \begin{itemize}
     \item Уровень сбережений $s_t$ индивида из поколения $t$ --- это результат решения оптимизационной задачи:
     \begin{align*}
  \max_{c_{1, t}, {c_{2, t+1}, s_t}} \ln(c_{1,t}) &+ \beta \ln(c_{2, t+1 })\\
  \intertext{при ограничениях:} 
  c_{1,t} + s_t &\leq w_t, \\
c_{2, t} &\leq R_{t+1} s_t.
\end{align*}
\item Условия первого порядка:
\begin{equation*}
    s_t = \frac{\beta}{1+\beta} w_t.
\end{equation*}
 \end{itemize}
 

\end{frame}

\subsection{Динамика капитала}
\begin{frame}
 \frametitle{\insertsection}
 \framesubtitle{\insertsubsection}
 \begin{itemize}
     \item Уровень капитала в момент времени $t+1$:
     \begin{equation*}
  K_{t+1} &= s_t L_t + (1-\delta) K_t.
  \end{equation*}
  \item Перейдём к капиталовооружённости на единицу труда:
  \begin{align*}
       k_{t+1} &= \frac{s_t}{1+n} + (1-\delta) \frac{k_t}{1+n};\\
  k_{t+1} &= \frac{\beta w_t}{(1+\beta)(1+n)} + (1-\delta) \frac{k_t}{1+n};\\
  k_{t+1} &=  \frac{\beta (1-\alpha) k_{t}^\alpha + (1-\delta)  k_t}{(1+\beta)(1+n)}.
  \end{align*}
 
  \item В устойчивом состоянии капиталовооружённость не меняется: $k^{*} = k_{t+1} = k_{t}$:
  \begin{equation*}
  k^{*} &= \left[\frac{\beta (1-\alpha)}{(1+\beta)(1+n) - (1-\delta)}\right]^{\left(\frac{1}{1-\alpha}\right)}
 \end{equation*}
\end{itemize}
 

\end{frame}

\subsection{Динамическая неэффективность}


\section{Модель 3-х поколений}
\subsection{Предпосылки}
\subsection{Решение}
\subsection{Симуляция}
% усложнение модели
% вечная молодость 
% эндогенная рождаемость
% многомерное население + симуляция


\end{document}