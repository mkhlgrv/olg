%!TEX TS-program = xelatex

\documentclass[c, dvipsnames]{beamer}  % [t], [c], или [b] --- вертикальное 
%\documentclass[handout, dvipsnames, c]{beamer} % Раздаточный материал (на слайдах всё сразу)
\input{preamble_slides}
\title[OLG-модели]{Модели пересекающихся поколений (OLG)}


\author[Михаил Гареев]{Михаил Гареев \\ \smallskip \scriptsize ЭО-15-01 \\ \smallskip \scriptsize \href{mailto:mkhlgrv@gmail.com}{\nolinkurl{mkhlgrv@gmail.com} }}

\superviser{к.э.н. Полбин А.В.}

\institute[РАНХиГС]{ \uppercase{
 Российская Академия Народного Хозяйства и  \\ Государственной Службы при Президенте Российской Федерации}}
\date{2020}


\titlegraphic{\includegraphics[scale=0.5]{logo/logo_ranepa.png}}
\titlegraphicii{\includegraphics[scale=0.5]{logo/logo_emit.png}}

\begin{document}

\frame[plain]{\titlepage}	% Титульный слайд


\section{Возникновение и основная идея}

\begin{frame}
 \frametitle{\insertsection}
    \begin{itemize}
        \item Проблема: модели экономического роста первого поколения (модель Солоу) предполагают экзогенную норму сбережения.
        \item Решение: модель, в которой сбережения формируются исходя из выбора экономических агентов:
        \begin{itemize}
        \item модель Рамсея --- Кааса --- Купманса (1963): бесконечно живущий агент выбирает уровень потребления и сбережения, максимизируя полезность;
        \item модель пересекающихся поколений Даймонда (1965): в каждый момент времени существует несколько поколений, молодые работают и сберегают, пожилые тратят свои сбережения. 
        \end{itemize}
    \end{itemize}
\end{frame}

\section{Базовая модель 2-х поколений}
\subsection{Предположения}
\begin{frame}
 \frametitle{\insertsection}
 \framesubtitle{\insertsubsection}
     \begin{itemize}
        \item Дискретное время;
        \item Индивиды живут два периода;
        \item Полезность индивида $U_t = \ln(c_{1,t}) + \beta \ln(c_{2, t+1 })$,
        \begin{itemize}
            \item $c_{1,t}$ --- потребление молодого поколения,
        \item $c_{2,t + 1}$ --- потребление пожилого поколения,
        \item $\beta \in (0, 1)$ --- коэффициент временных предпочтений,
        \end{itemize}
        \item Численность населения растёт с постоянным темпом $n$: $L_t = (1+n)^t L_0$;
        \item Индивиды работают только в первый период жизни, неэластично предлагая единицу труда и получая зарплату $w_t$.
        
    \end{itemize}

\end{frame}

\begin{frame}
 \frametitle{\insertsection}
 \framesubtitle{\insertsubsection}
     \begin{itemize}
     \item Выпуск описывается функцией Кобба --- Дугласа: $Y_t =F(K_t, L_t) = {K_t}^\alpha {L_t}^{1-\alpha}$;
        \item Рынки факторов производства находятся в совершенной конкуренции;
        \item Обозначим $k_t = K_t/L_t$, $f(k_t) = F(K_t, 1)$, тогда рентная цена капитала, равная доходности сбережений $R_t = 1-\delta + f^{'}(k_t)$, где $\delta$ --- амортизация;
        \item Зарплата $w_t = f(k_t) - R_t k_t$;
        \item Единица сбережений $s_t$ без потерь трансформируется в единицу капитала $K_t$.
    \end{itemize}

\end{frame}

\subsection{Задача индивида}
\begin{frame}
 \frametitle{\insertsection}
 \framesubtitle{\insertsubsection}
 \begin{itemize}
     \item Уровень сбережений $s_t$ индивида из поколения $t$ --- это результат решения оптимизационной задачи:
     \begin{align*}
  \max_{c_{1, t}, {c_{2, t+1}, s_t}} \ln(c_{1,t}) &+ \beta \ln(c_{2, t+1 })\\
  \intertext{при ограничениях:} 
  c_{1,t} + s_t &\leq w_t, \\
c_{2, t} &\leq R_{t+1} s_t.
\end{align*}
\item Условия первого порядка:
\begin{equation*}
    s_t = \frac{\beta}{1+\beta} w_t.
\end{equation*}
 \end{itemize}
 

\end{frame}

\subsection{Динамика капитала}
\begin{frame}
 \frametitle{\insertsection}
 \framesubtitle{\insertsubsection}
 \begin{itemize}
     \item Уровень капитала в момент времени $t+1$:
     \begin{equation*}
  K_{t+1} &= s_t L_t + (1-\delta) K_t.
  \end{equation*}
  \item Капиталовооружённость единицы труда:
  \begin{align*}
       k_{t+1} &= \frac{s_t}{1+n} + (1-\delta) \frac{k_t}{1+n};\\
  k_{t+1} &= \frac{\beta w_t}{(1+\beta)(1+n)} + (1-\delta) \frac{k_t}{1+n};\\
  k_{t+1} &=  \frac{\beta (1-\alpha) k_{t}^\alpha + (1-\delta)  k_t}{(1+\beta)(1+n)}.
  \end{align*}
 
  \item В устойчивом состоянии капиталовооружённость не меняется: $k^{*} = k_{t+1} = k_{t}$:
  \begin{equation*}
  k^{*} &= \left[\frac{\beta (1-\alpha)}{(1+\beta)(1+n) - (1-\delta)}\right]^{\left(\frac{1}{1-\alpha}\right)}
 \end{equation*}
\end{itemize}
 

\end{frame}

% \subsection{Динамическая неэффективность}


\section{Модель 3-х поколений}
\subsection{Предположения}
\begin{frame}[shrink=5]

 \frametitle{\insertsection}
 \framesubtitle{\insertsubsection}
     \begin{itemize}
        \item Дискретное время;
        \item Индивиды живут три периода;
        \item Полезность индивида из поколения $\tau$ $U_\tau = \ln(c_{\tau,\tau}) + \beta \ln(c_{\tau, \tau+1 })+\beta^2\ln(c_{\tau, \tau+2 }) $, где:
        \begin{itemize}
        \item $c_{\tau,\tau} \equiv c_\tau^y $ --- потребление молодого поколения в период времени $\tau$,
        \item $c_{\tau,\tau+1} \equiv c_{\tau+1}t^m$ --- потребление среднего поколения в период времени $\tau+1$,
        \item $c_{\tau,\tau+2} \equiv c_{\tau+2}t^o$ --- потребление пожилого поколения в период времени $\tau+2$,
        \item $\beta \in (0, 1)$ --- коэффициент временных предпочтений;
        \end{itemize}
        \item Численность поколения $\tau$ в момент времени $t$:
        \begin{equation}
            N_{\tau,t} =  \begin{cases}
            (1+n)^\tau N_0,& \text{если } \tau \leq t + 2,\\
            0, & \text{если } \tau > t+ 2;
        \end{cases}
        \end{equation}
        \item Численность населения в момент времени $t$: 
        \begin{equation}
            \widetilde{N_t} = N_{t-2, t} + N_{t-1, t}+ N{t, t} = (1+n)^{t-2}N_0 + (1+n)^{t-1}N_0+(1+n)^t N_0.
        \end{equation}
        
    \end{itemize}
    
    

\end{frame}

\begin{frame}
 \frametitle{\insertsection}
 \framesubtitle{\insertsubsection}
     \begin{itemize}
     \item Индивиды работают два первых периода жизни: молодому поколению доступна единица труда, среднему поколению доступно $d>0$ труда;
        \item Предложение труда неэластично. Индивиды предлагают весь доступный труд и получают зарплату $w_t$: 
        \begin{equation}
            L_t =  N_{t-1, t}+ N{t, t} = (1+n)^{t-1}N_0+(1+n)^t N_0;
        \end{equation}
     \item Выпуск описывается функцией Кобба-Дугласа: $Y_t =F(K_t, L_t) = {K_t}^\alpha {L_t}^{1-\alpha}$;
        \item Рынки факторов производства находятся в совершенной конкуренции;
        \item Рентная цена капитала $R_t = 1-\delta + f^{'}(k_t)$;
        \item Зарплата $w_t = f(k_t) - R_t k_t$;
        \item Единица сбережений $s_t$ без потерь трансформируется в единицу капитала $K_t$.
    \end{itemize}

\end{frame}

\subsection{Задача индивида}
\begin{frame}
 \frametitle{\insertsection}
 \framesubtitle{\insertsubsection}
 \begin{itemize}
     \item Уровень сбережений $s_{\tau, t}$ индивида из поколения $\tau$ в момент времени $t$ --- это результат решения оптимизационной задачи:
     \begin{align*}
  \max_{c_{1, t}, {c_{2, t+1}, s_t}} &
\ln(c_{\tau,\tau}) + \beta \ln(c_{\tau, \tau+1 })+\beta^2\ln(c_{\tau, \tau+2 })\\
\intertext{при ограничениях:} 
    c_{\tau,\tau} + s_{\tau,\tau} &\leq w_\tau, \\
    c_{\tau,\tau+1} + s_{\tau,\tau+1} &\leq d w_{\tau+1} + R_{\tau+1} s_{\tau,\tau},\\
    c_{\tau,\tau+2} &\leq R_{t+2} s_{\tau,\tau+1}.
\end{align*}
\item Условия первого порядка:
\begin{equation*}
    s_t = \frac{\beta}{1+\beta} w_t.
\end{equation*}
 \end{itemize}
\end{frame}


\subsection{Накопление капитала}
\begin{frame}
 \frametitle{\insertsection}
 \framesubtitle{\insertsubsection}
 \begin{itemize}
     \item Уровень капиталовооружённости в момент времени $t+1$:
\begin{align*}
k_{t+1}& = \frac{c_t^y}{1+n+d}+ \frac{c_t^m}{(1+n)(1+n+d)} + k_t \frac{1-\delta}{1+n},\\
\intertext{где: }
c_t^y &= w_t (1+\beta) \beta - \frac{w_{t+1} d}{R_{t+1} (1+\beta+\beta^2)},\\
 s_t^m &= \frac{\beta^2(d w_t + (w_{t-1} R_t)) - w_t d }{1+\beta+\beta^2}.
\end{align*}
 \end{itemize}
\end{frame}

\subsection{Решение}
\begin{frame}
 \frametitle{\insertsection}
 \framesubtitle{\insertsubsection}
\begin{itemize}
    \item Аналитическое решение невозможно, приходится использовать численные методы.
    \item Алгоритм:
    \begin{enumerate}
                \item Численно находится капиталовооружённость в устойчивом состоянии $k^{*}: k_{t+1} = k_t$;
    \item Находится первое приближение (линейный тренд от $k_0$ до $k_T = k^{*}$);
    \item Для каждого момента времени $t \in (0, T)$ при известных капитале и труде рассчитываются цены факторов производства $R_t, w_t$;
    \item При известных ценах на факторы рассчитываются потребление и сбережения;
    \item При известных сбережениях формируется новая траектория капиталовооружённости, которая прибавляется к предыдущей с шагом $\eta$;
    \item Процедура повторяется с шага 3, пока разница между моделями не окажется меньше заданного уровня $\zeta$.
    \end{enumerate}

    
\end{itemize}
\end{frame}

\begin{frame}
 \frametitle{\insertsection}
 \framesubtitle{\insertsubsection}

\begin{figure}
    \centering
    \includegraphics{}
    \caption{Caption}
    \label{fig:my_label}
\end{figure}
\end{frame}


\subsection{Симуляция}
\begin{frame}
 \frametitle{\insertsection}
 \framesubtitle{\insertsubsection}

\end{frame}
\subsection{Планы}
\begin{frame}
 \frametitle{\insertsection}
 \framesubtitle{\insertsubsection}
\begin{itemize}
    \item Добавление налогов;
    \item Гетерогенные поколения;
    \item Увеличение числа поколений;
    \item Добавление досуга в функцию полезности, эластичное предложение труда, моделирование изменения пенсионного возраста;
    \item Открытая экономика.
\end{itemize}
\end{frame}

\end{document}