%!TEX TS-program = xelatex
\documentclass[12pt,a4paper,oneside]{article}


%%%%%%%%%% Математика %%%%%%%%%%
\usepackage{amsmath,amsfonts,amssymb,amsthm,mathtools}
\usepackage{upgreek}
\usepackage{hyphenat}

\usepackage[T4,T1]{fontenc}
\makeatletter
\newcommand{\do@openE}[1]{%
  \mbox{\fontsize{#1}\z@\usefont{T4}{cmr}{m}{n}\symbol{130}}%
}
\newcommand{\openE}{\mathord{\mathchoice
  {\do@openE\tf@size}
  {\do@openE\tf@size}
  {\do@openE\sf@size}
  {\do@openE\ssf@size}
}}
\makeatother




%%%%%%%%%% Шрифты %%%%%%%%
\usepackage[english, russian]{babel} % выбор языка для документа
\usepackage[utf8]{inputenc} % задание utf8 кодировки исходного tex файла
\usepackage[X2,T2A]{fontenc} % кодировка

\usepackage{fontspec} % пакет для подгрузки шрифтов
%\setmainfont{Times New Roman} % задаёт основной шрифт документа

\setmainfont[
  Ligatures=TeX,
  Extension=.otf,
  BoldFont=cmunbx,
  ItalicFont=cmunti,
  BoldItalicFont=cmunbi,
]{cmunrm}


\usepackage{unicode-math} % пакет для установки математического шрифта
\setmathfont{Asana-Math.otf} % шрифт для математики



%%%%%%%%%% Работа с картинками %%%%%%%%%
\usepackage{graphicx} % Для вставки рисунков
\usepackage{graphics}
\graphicspath{{images/}{pictures/}} % можно указать папки с картинками
\usepackage{wrapfig} % Обтекание рисунков и таблиц текстом

%%%%%%%%%% Работа с таблицами %%%%%%%%%%
\usepackage{tabularx} % новые типы колонок
\usepackage{tabulary} % и ещё новые типы колонок
\usepackage{array,delarray} % Дополнительная работа с таблицами
\usepackage{longtable} % Длинные таблицы
\usepackage{multirow} % Слияние строк в таблице
\usepackage{float} % возможность позиционировать объекты в нужном месте

\usepackage{booktabs} % таблицы как в книгах
% Заповеди из документации к booktabs:
% 1. Будь проще! Глазам должно быть комфортно
% 2. Не используйте вертикальные линни
% 3. Не используйте двойные линии. Как правило, достаточно трёх горизонтальных линий
% 4. Единицы измерения - в шапку таблицы
% 5. Не сокращайте .1 вместо 0.1
% 6. Повторяющееся значение повторяйте, а не говорите "то же"
% 7. Есть сомнения? Выравнивай по левому краю!

% вычисляемые колонки по tabularx
\newcolumntype{C}{>{\centering\arraybackslash}X}
\newcolumntype{L}{>{\raggedright\arraybackslash}X}
\newcolumntype{Y}{>{\arraybackslash}X}
\newcolumntype{Z}{>{\centering\arraybackslash}X}

%%%%%%%%%% Графика и рисование %%%%%%%%%%
\usepackage{tikz} % язык для рисования графики из latex'a

%%%%%%%%%% Гиперссылки %%%%%%%%%%
\usepackage{xcolor} % разные цвета

\usepackage{hyperref}
\hypersetup{
unicode=true, % позволяет использовать юникодные символы
colorlinks=true, % true - цветные ссылки, false - ссылки в рамках
urlcolor =black, % цвет ссылки на url
linkcolor=black, % внутренние ссылки
citecolor=black, % на библиографию
breaklinks % если ссылка не умещается в одну строку, разбивать ли ее на две части?
}

%%%%%%%%%% Другие приятные пакеты %%%%%%%%%
\usepackage{multicol} % несколько колонок
\usepackage{verbatim} % для многострочных комментариев
\usepackage{cmap} % для кодировки шрифтов в pdf

\usepackage{enumitem} % дополнительные плюшки для списков
% например \begin{Enumerate}[resume] позволяет продолжить нумерацию в новом списке

% \usepackage{todonotes} % для вставки в документ заметок о том, что осталось сделать
% \todo{Здесь надо коэффициенты исправить}
% \missingfigure{Здесь будет Последний день Помпеи}
%
 
%\listoftodos —- печатает все поставленные \todo'шки

%%%%%%%%%% Список литературы %%%%%%%%%%

 \usepackage[language=english,backend=biber,giveninits=true,style=authoryear]{biblatex}
%\usepackage[language=russian,backend=biber,giveninits=true,style=authoryear]{biblatex}
\renewcommand*{\nameyeardelim}{\addcomma\space}
\addbibresource{bib_en.bib}
% \addbibresource{bib.bib}
\renewbibmacro{in:}{}

\makeatletter
\AtEveryBibitem{\global\undef\bbx@lasthash}
\makeatother



\usepackage[title,titletoc]{appendix}

% \DeclareFieldFormat{labelnumberwidth}{#1)}

\usepackage{titling}


%%%%%%% Geometry %%%%%%%%
\usepackage{geometry}
\geometry{ a4paper,
right = 20mm,
 left= 40mm,
 top=20mm,
 bottom = 20mm
 }
 
 
 %%% Заголовки
\usepackage[indentfirst]{titlesec}{\raggedleft}


\usepackage{longtable}

\usepackage{animate}[2017/05/18]
\usepackage{graphicx}
% Цветные комментарии
% \usepackage{todonotes}
\usepackage[disable]{todonotes}
\reversemarginpar
\setlength{\marginparwidth}{3.5cm}
